% Metódy inžinierskej práce

\documentclass[10pt,twoside,slovak,a4paper]{article}

\usepackage[slovak]{babel}
%\usepackage[T1]{fontenc}
\usepackage[IL2]{fontenc} % lepšia sadzba písmena Ľ než v T1
\usepackage[utf8]{inputenc}
\usepackage{graphicx}
\usepackage{url} % príkaz \url na formátovanie URL
\usepackage{hyperref} % odkazy v texte budú aktívne (pri niektorých triedach dokumentov spôsobuje posun textu)

\usepackage{cite}
%\usepackage{times}

\pagestyle{headings}

\title{Výskum a vývoj inteligentného asistenta pre akademickú podporu študentov\\
\large EDUAI
\thanks{Semestrálny projekt v predmete Metódy inžinierskej práce, ak. rok 2025/26, vedenie: Ing. Ivan Kapustík}} % meno a priezvisko vyučujúceho na cvičeniach'

\author{Volodymyr Kutserubov, Matej Laurinec, Martin Lazar\\[2pt]
	{\small Slovenská technická univerzita v Bratislave}\\
	{\small Fakulta informatiky a informačných technológií}\\
	{\small \texttt{xkutserubov@stuba.sk, xlaurinec@stuba.sk, xlazar@stuba.sk}}
	}

\date{\small 10.  oktober 2025} % upravte

%testtesteefpjerofjwelkfjewlkfjewlkfjew

\begin{document}

\maketitle

\begin{abstract}
Projekt EDUAI je zameraný na výskum a vývoj inteligentného asistenta „EduGuide AI“, ktorý využíva metódy umelej inteligencie a spracovania prirodzeného jazyka (NLP) na automatizovanú akademickú podporu študentov vysokých škôl. Cieľom je preskúmať, ako možno moderné jazykové modely (napr. BERT, LLaMA, GPT) adaptovať pre spracovanie akademicky špecifických dát v slovenčine a angličtine a vytvoriť prototyp asistenta schopného odpovedať na otázky z univerzitných informačných systémov, študijných poriadkov a interných dokumentov.

Projekt je aktuálny a originálny svojím zameraním na adaptáciu AI modelov v akademickom prostredí, čo je v rámci slovenského výskumného prostredia inovatívny prístup. Výsledkom bude výskumný prototyp asistenta s preukázateľne merateľnou presnosťou odpovedí a metodika hodnotenia kvality AI interakcie v akademickom kontexte.

Očakávané výstupy zahŕňajú:
\begin{itemize}
\item Metodiku trénovania a evaluácie univerzitne špecializovaných NLP modelov v slovenčine a angličtine.
\item Experimentálne výsledky o efektívnosti AI asistenta pri znižovaní administratívnej záťaže.
\item Prototyp systému EduGuide AI ako dôkaz konceptu.
\item Odporúčania pre integráciu AI technológií do univerzitných informačných systémov.
\end{itemize}

Projekt prinesie spoločenský a ekonomický dopad – zlepší dostupnosť informácií pre študentov, zvýši efektivitu akademickej komunikácie a prispeje k digitalizácii vysokoškolského prostredia. Zároveň vytvorí výskumný základ pre ďalšie inovácie v oblasti AI pre vzdelávanie a posilní postavenie Slovenska v európskom výskumnom priestore inteligentných systémov.
\end{abstract}

\section{Úvod}

Motivujte čitateľa a vysvetlite, o čom píšete. Úvod sa väčšinou nedelí na časti.

Uveďte explicitne štruktúru článku. Tu je nejaký príklad.
Základný problém, ktorý bol naznačený v úvode, je podrobnejšie vysvetlený v časti~\ref{nejaka}.
Dôležité súvislosti sú uvedené v častiach~\ref{dolezita} a~\ref{dolezitejsia}.
Záverečné poznámky prináša časť~\ref{zaver}.



\section{Nejaká časť} \label{nejaka}

Z obr.~\ref{f:rozhod} je všetko jasné. 

\begin{figure*}[tbh]
\centering
%\includegraphics[scale=1.0]{diagram.pdf}
Aj text môže byť prezentovaný ako obrázok. Stane sa z neho označný plávajúci objekt. Po vytvorení diagramu zrušte znak \texttt{\%} pred príkazom \verb|\includegraphics| označte tento riadok ako komentár (tiež pomocou znaku \texttt{\%}).
\caption{Rozhodujúci argument.}
\label{f:rozhod}
\end{figure*}



\section{Iná časť} \label{ina}

Základným problémom je teda\ldots{} Najprv sa pozrieme na nejaké vysvetlenie (časť~\ref{ina:nejake}), a potom na ešte nejaké (časť~\ref{ina:nejake}).\footnote{Niekedy môžete potrebovať aj poznámku pod čiarou.}

Môže sa zdať, že problém vlastne nejestvuje\cite{Coplien:MPD}, ale bolo dokázané, že to tak nie je~\cite{Czarnecki:Staged, Czarnecki:Progress}. Napriek tomu, aj dnes na webe narazíme na všelijaké pochybné názory\cite{PLP-Framework}. Dôležité veci možno \emph{zdôrazniť kurzívou}.


\subsection{Nejaké vysvetlenie} \label{ina:nejake}

Niekedy treba uviesť zoznam:

\begin{itemize}
\item jedna vec
\item druhá vec
	\begin{itemize}
	\item x
	\item y
	\end{itemize}
\end{itemize}

Ten istý zoznam, len číslovaný:

\begin{enumerate}
\item jedna vec
\item druhá vec
	\begin{enumerate}
	\item x
	\item y
	\end{enumerate}
\end{enumerate}


\subsection{Ešte nejaké vysvetlenie} \label{ina:este}

\paragraph{Veľmi dôležitá poznámka.}
Niekedy je potrebné nadpisom označiť odsek. Text pokračuje hneď za nadpisom.



\section{Dôležitá časť} \label{dolezita}




\section{Ešte dôležitejšia časť} \label{dolezitejsia}




\section{Záver} \label{zaver} % prípadne iný variant názvu


%\acknowledgement{Ak niekomu chcete poďakovať\ldots}


% týmto sa generuje zoznam literatúry z obsahu súboru literatura.bib podľa toho, na čo sa v článku odkazujete
\bibliography{literatura}
\bibliographystyle{plain} % prípadne alpha, abbrv alebo hociktorý iný
\end{document}
