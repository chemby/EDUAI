% Metódy inžinierskej práce

\documentclass[10pt,twoside,slovak,a4paper]{article}

\usepackage[slovak]{babel}
%\usepackage[T1]{fontenc}
\usepackage[IL2]{fontenc} % lepšia sadzba písmena Ľ než v T1
\usepackage[utf8]{inputenc}
\usepackage{graphicx}
\usepackage{url} % príkaz \url na formátovanie URL
\usepackage{hyperref} % odkazy v texte budú aktívne (pri niektorých triedach dokumentov spôsobuje posun textu)

\usepackage{cite}
%\usepackage{times}

\pagestyle{headings}

\title{Výskum a vývoj inteligentného asistenta pre akademickú podporu študentov\\
\large EDUAI
\thanks{Semestrálny projekt v predmete Metódy inžinierskej práce, ak. rok 2025/26, vedenie: Ing. Ivan Kapustík}} % meno a priezvisko vyučujúceho na cvičeniach'

\author{Volodymyr Kutserubov, Matej Laurinec, Martin Lazar\\[2pt]
	{\small Slovenská technická univerzita v Bratislave}\\
	{\small Fakulta informatiky a informačných technológií}\\
	{\small \texttt{xkutserubov@stuba.sk, xlaurinec@stuba.sk, xlazar@stuba.sk}}
	}

\date{\small 10.  oktober 2025} % upravte



\begin{document}

\maketitle

\begin{abstract}
Projekt EDUAI je zameraný na výskum a vývoj inteligentného asistenta „EduGuide AI“, ktorý využíva metódy umelej inteligencie a spracovania prirodzeného jazyka (NLP) na automatizovanú akademickú podporu študentov vysokých škôl. Cieľom je preskúmať, ako možno moderné jazykové modely (napr. BERT, LLaMA, GPT) adaptovať pre spracovanie akademicky špecifických dát v slovenčine a angličtine a vytvoriť prototyp asistenta schopného odpovedať na otázky z univerzitných informačných systémov, študijných poriadkov a interných dokumentov.

Projekt je aktuálny a originálny svojím zameraním na adaptáciu AI modelov v akademickom prostredí, čo je v rámci slovenského výskumného prostredia inovatívny prístup. Výsledkom bude výskumný prototyp asistenta s preukázateľne merateľnou presnosťou odpovedí a metodika hodnotenia kvality AI interakcie v akademickom kontexte.

Očakávané výstupy zahŕňajú:
\begin{itemize}
\item Metodiku trénovania a evaluácie univerzitne špecializovaných NLP modelov v slovenčine a angličtine.
\item Experimentálne výsledky o efektívnosti AI asistenta pri znižovaní administratívnej záťaže.
\item Prototyp systému EduGuide AI ako dôkaz konceptu.
\item Odporúčania pre integráciu AI technológií do univerzitných informačných systémov.
\end{itemize}

Projekt prinesie spoločenský a ekonomický dopad – zlepší dostupnosť informácií pre študentov, zvýši efektivitu akademickej komunikácie a prispeje k digitalizácii vysokoškolského prostredia. Zároveň vytvorí výskumný základ pre ďalšie inovácie v oblasti AI pre vzdelávanie a posilní postavenie Slovenska v európskom výskumnom priestore inteligentných systémov.
\end{abstract}
\tableofcontents
\newpage 

\section{Úvod a teoretické východiská}

Štúdium na vysokej škole dnes nie je úplne iné než kedysi, no požiadavky na študentov i školu sa výrazne zvýšili. Študenti musia spracovať veľké množstvo informácií, orientovať sa v rozvrhoch, predpisoch a interných dokumentoch fakulty či univerzity. Tradičné spôsoby poskytovania informácií, ako sú PDF súbory alebo webové portály, často vyžadujú veľa času a môžu pôsobiť neprehľadne. To zvyšuje administratívnu záťaž študentov a zároveň znižuje efektívnosť ich štúdia a komunikácie s učiteľmi a školou.

V poslednej dobe sa ukazuje, že študentom môžu pomôcť inteligentní asistenti založení na umelej inteligencii (AI). Takíto asistenti, hlavne pokročilé modely spracovania prirodzeného jazyka (NLP), môžu študentom rýchlo a presne odpovedať na otázky — napríklad kedy sú skúšky, čo všetko je potrebné na absolvovanie predmetu alebo aké sú interné pravidlá školy. Projekt EDUAI sa preto zameriava na vývoj asistenta „EduGuide AI“, ktorý využíva modely ako BERT, LLaMA či GPT a je prispôsobený pre univerzitné prostredie v slovenčine aj angličtine. Cieľom je adaptovať tieto modely na spracovanie univerzitne špecifických dát a vytvoriť funkčný prototyp, ktorý študentom uľahčí prístup k relevantným informáciám a zníži administratívnu záťaž.

\subsection{Stav poznania}

V posledných rokoch sa výrazne rozšíril výskum AI vo vzdelávaní (AIEd). Napríklad meta-systematická recenzia zameraná na vyššie vzdelávanie ukazuje, že existuje mnoho štúdií o využití AI v tomto sektore, pričom autori upozorňujú na potrebu väčšej etickej, metodologickej a kontextuálnej reflexie pri budúcom výskume \cite{bond2024aihed}. Ďalší systematický prehľad zdôrazňuje príležitosti, výzvy a smerovanie budúceho výskumu v AIEd a upozorňuje na problémy, ako sú jazykové bariéry, nedostatok kvalitných dát či nedostatočná integrácia modelov do akademického prostredia \cite{chen2022ai, khosravi2023ai}.

Napriek pokroku väčšina existujúcich modelov nie je optimalizovaná pre slovenský jazyk ani pre špecifické potreby univerzít, napríklad interné dokumenty a študijné plány. Modely ako BERT alebo GPT sú trénované hlavne na všeobecných dátach a často nedokážu presne odpovedať na otázky týkajúce sa akademických pravidiel alebo štruktúrovaných údajov školy. Projekt EDUAI preto identifikuje medzeru vo výskume — je potrebné adaptovať jazykové modely na akademické texty v slovenčine a angličtine.

Ďalším dôležitým aspektom je viacjazyčná podpora. Väčšina existujúcich nástrojov funguje iba v angličtine, čo môže byť limitujúce pre slovenské prostredie. Adaptácia modelu na slovenčinu vyžaduje techniky ako transfer learning a fine-tuning na lokálnych dátach, s následným experimentálnym overením kvality odpovedí v akademickom kontexte.

Na záver možno povedať, že projekt EDUAI sa snaží spojiť tri kľúčové veci: kvalitné jazykové modely, akademický obsah v slovenčine a angličtine a praktický prototyp asistenta. Takéto spojenie môže výrazne zlepšiť orientáciu študentov v univerzitných informáciách a zvýšiť efektivitu komunikácie so školou.

%\acknowledgement{Ak niekomu chcete poďakovať\ldots}


% týmto sa generuje zoznam literatúry z obsahu súboru literatura.bib podľa toho, na čo sa v článku odkazujete
\bibliography{literatura}
\bibliographystyle{plain} % prípadne alpha, abbrv alebo hociktorý iný
\end{document}
