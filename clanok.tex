% Metódy inžinierskej práce

\documentclass[10pt,twoside,slovak,a4paper]{article}

\usepackage[slovak]{babel}
%\usepackage[T1]{fontenc}
\usepackage[IL2]{fontenc} % lepšia sadzba písmena Ľ než v T1
\usepackage[utf8]{inputenc}
\usepackage{graphicx}
\usepackage{url} % príkaz \url na formátovanie URL
\usepackage{hyperref} % odkazy v texte budú aktívne (pri niektorých triedach dokumentov spôsobuje posun textu)

\usepackage{cite}
%\usepackage{times}

\pagestyle{headings}

\title{Výskum a vývoj inteligentného asistenta pre akademickú podporu študentov\\
\large EDUAI
\thanks{Semestrálny projekt v predmete Metódy inžinierskej práce, ak. rok 2025/26, vedenie: Ing. Ivan Kapustík}} % meno a priezvisko vyučujúceho na cvičeniach'

\author{Volodymyr Kutserubov, Matej Laurinec, Martin Lazar\\[2pt]
	{\small Slovenská technická univerzita v Bratislave}\\
	{\small Fakulta informatiky a informačných technológií}\\
	{\small \texttt{xkutserubov@stuba.sk, xlaurinec@stuba.sk, xlazar@stuba.sk}}
	}

\date{\small 10.  oktober 2025} % upravte



\begin{document}

\maketitle

\begin{abstract}
Projekt EDUAI je zameraný na výskum a vývoj inteligentného asistenta „EduGuide AI“, ktorý využíva metódy umelej inteligencie a spracovania prirodzeného jazyka (NLP) na automatizovanú akademickú podporu študentov vysokých škôl. Cieľom je preskúmať, ako možno moderné jazykové modely (napr. BERT, LLaMA, GPT) adaptovať pre spracovanie akademicky špecifických dát v slovenčine a angličtine a vytvoriť prototyp asistenta schopného odpovedať na otázky z univerzitných informačných systémov, študijných poriadkov a interných dokumentov.

Projekt je aktuálny a originálny svojím zameraním na adaptáciu AI modelov v akademickom prostredí, čo je v rámci slovenského výskumného prostredia inovatívny prístup. Výsledkom bude výskumný prototyp asistenta s preukázateľne merateľnou presnosťou odpovedí a metodika hodnotenia kvality AI interakcie v akademickom kontexte.

Očakávané výstupy zahŕňajú:
\begin{itemize}
\item Metodiku trénovania a evaluácie univerzitne špecializovaných NLP modelov v slovenčine a angličtine.
\item Experimentálne výsledky o efektívnosti AI asistenta pri znižovaní administratívnej záťaže.
\item Prototyp systému EduGuide AI ako dôkaz konceptu.
\item Odporúčania pre integráciu AI technológií do univerzitných informačných systémov.
\end{itemize}

Projekt prinesie spoločenský a ekonomický dopad – zlepší dostupnosť informácií pre študentov, zvýši efektivitu akademickej komunikácie a prispeje k digitalizácii vysokoškolského prostredia. Zároveň vytvorí výskumný základ pre ďalšie inovácie v oblasti AI pre vzdelávanie a posilní postavenie Slovenska v európskom výskumnom priestore inteligentných systémov.
\end{abstract}


%\acknowledgement{Ak niekomu chcete poďakovať\ldots}


% týmto sa generuje zoznam literatúry z obsahu súboru literatura.bib podľa toho, na čo sa v článku odkazujete
\bibliography{literatura}
\bibliographystyle{plain} % prípadne alpha, abbrv alebo hociktorý iný
\end{document}
