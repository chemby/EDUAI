% Metódy inžinierskej práce

\documentclass[10pt,twoside,slovak,a4paper]{article}

\usepackage[slovak]{babel}
%\usepackage[T1]{fontenc}
\usepackage[IL2]{fontenc} % lepšia sadzba písmena Ľ než v T1
\usepackage[utf8]{inputenc}
\usepackage{graphicx}
\usepackage{url} % príkaz \url na formátovanie URL
\usepackage{hyperref} % odkazy v texte budú aktívne (pri niektorých triedach dokumentov spôsobuje posun textu)

\usepackage{cite}
%\usepackage{times}

\pagestyle{headings}

\title{Výskum a vývoj inteligentného asistenta pre akademickú podporu študentov\\
\large EDUAI
\thanks{Semestrálny projekt v predmete Metódy inžinierskej práce, ak. rok 2025/26, vedenie: Ing. Ivan Kapustík}} % meno a priezvisko vyučujúceho na cvičeniach'

\author{Volodymyr Kutserubov, Matej Laurinec, Martin Lazar\\[2pt]
	{\small Slovenská technická univerzita v Bratislave}\\
	{\small Fakulta informatiky a informačných technológií}\\
	{\small \texttt{xkutserubov@stuba.sk, xlaurinec@stuba.sk, xlazar@stuba.sk}}
	}

\date{\small 10.  oktober 2025} % upravte


\begin{document}

\maketitle

\begin{abstract}
Projekt EDUAI je zameraný na výskum a vývoj inteligentného asistenta „EduGuide AI“, ktorý využíva metódy umelej inteligencie a spracovania prirodzeného jazyka (NLP) na automatizovanú akademickú podporu študentov vysokých škôl. Cieľom je preskúmať, ako možno moderné jazykové modely (napr. BERT, LLaMA, GPT) adaptovať pre spracovanie akademicky špecifických dát v slovenčine a angličtine a vytvoriť prototyp asistenta schopného odpovedať na otázky z univerzitných informačných systémov, študijných poriadkov a interných dokumentov.

Projekt je aktuálny a originálny svojím zameraním na adaptáciu AI modelov v akademickom prostredí, čo je v rámci slovenského výskumného prostredia inovatívny prístup. Výsledkom bude výskumný prototyp asistenta s preukázateľne merateľnou presnosťou odpovedí a metodika hodnotenia kvality AI interakcie v akademickom kontexte.

Očakávané výstupy zahŕňajú:
\begin{itemize}
\item Metodiku trénovania a evaluácie univerzitne špecializovaných NLP modelov v slovenčine a angličtine.
\item Experimentálne výsledky o efektívnosti AI asistenta pri znižovaní administratívnej záťaže.
\item Prototyp systému EduGuide AI ako dôkaz konceptu.
\item Odporúčania pre integráciu AI technológií do univerzitných informačných systémov.
\end{itemize}

Projekt prinesie spoločenský a ekonomický dopad – zlepší dostupnosť informácií pre študentov, zvýši efektivitu akademickej komunikácie a prispeje k digitalizácii vysokoškolského prostredia. Zároveň vytvorí výskumný základ pre ďalšie inovácie v oblasti AI pre vzdelávanie a posilní postavenie Slovenska v európskom výskumnom priestore inteligentných systémov.
\end{abstract}

\section{Úvod}

1) Kontext a motivácia výskumu...\\
2) Ciele a výskumné otázky...\\
3) Štruktúra práce...\\



\section{Teoretické základy a súčasný stav výskumu} \label{teoreticke_zaklady}

\subsection{Umelá inteligencia v akademickom prostredí} \label{ai_v_akad_prostredi}
Prehľad využívania AI vo vzdelávaní a existujúcich prístupov (ChatGPT Edu, Khanmigo, Google LearnLM).

[TABUĽKA: Porovnanie existujúcich akademických AI asistentov]
\subsection{NLP technológie a jazykové modely} \label{nlp_modely}
Popis základov spracovania prirodzeného jazyka, konceptov embeddingov, transformerov a fine-tuningu.

[DIAGRAM: Schéma architektúry Transformeru]
\subsection{Slovenské a viacjazyčné NLP modely} \label{slovenske_nlp_modely}
Prehľad dostupných jazykových modelov pre slovenčinu (napr. SlovBERT, mBERT, LLaMA-3 multilingual).

[GRAF: Výkonnosť slovenských NLP modelov v benchmarkoch (F1, BLEU)]
\subsection{Etické a spoločenské aspekty použitia AI} \label{eticke_aspekty}
Diskusia o transparentnosti, zaujatosti modelov, ochrane osobných údajov a etických normách v akademickom prostredí.

[DIAGRAM: Etický rámec AI v akademickej komunikácii]


\section{Metodológia výskumu} \label{metodologia}

\subsection{Výskumný dizajn} \label{dizajn}
Popis výskumného prístupu (experimentálny, kvantitatívny, vývojový). Definícia závislých a nezávislých premenných.

[DIAGRAM: Schéma výskumného rámca projektu EDUAI]
\subsection{Zber a spracovanie dát} \label{zber_dat}
Opis zdrojov dát (interné univerzitné dokumenty, študijné poriadky, FAQ, učebné texty) a ich predspracovania (tokenizácia, normalizácia, anotácia).

[GRAF: Distribúcia dát podľa typu dokumentu]
\subsection{Tréning a adaptácia modelov} \label{trening}
Detailný opis procesu fine-tuningu jazykových modelov (BERT, LLaMA, GPT) na akademických dátach.

[DIAGRAM: Pipeline trénovania EduGuide AI – od dát po model]
\subsection{Evaluácia a metriky} \label{evaluacia}
Popis metrík hodnotenia presnosti a kvality odpovedí (Accuracy, BLEU, ROUGE, F1-score, Human Evaluation).

[GRAF: Porovnanie metrík pre rôzne modely]
\subsection{Validácia a experimentálny plán} \label{validacia}
Metódy testovania v reálnych scenároch (dotazy študentov, akademické otázky).

[OBRÁZOK: Ukážka testovacieho prostredia EduGuide AI]


\section{Implementácia systému EduGuide AI} \label{implementacia}

\subsection{Architektúra systému} \label{architektura}
Prehľad komponentov: NLP modul, databázová vrstva, API rozhranie, používateľské rozhranie.

[DIAGRAM: Architektúra systému EduGuide AI]
\subsection{Integrácia s univerzitnými systémami} \label{integracia}
Popis spôsobu prístupu k univerzitným informačným systémom (API, databázové dotazy, autentifikácia).

[DIAGRAM: Tok dát medzi EduGuide AI a univerzitným systémom]
\subsection{Používateľské rozhranie a UX návrh} \label{ux}
Opis rozhrania asistenta (chatové okno, vyhľadávacie funkcie, jazyková voľba).

[OBRÁZOK: Screenshot návrhu používateľského rozhrania prototypu]
\subsection{Technologický stack a implementačné nástroje} \label{tech_stack}
Prehľad použitých technológií (Python, PyTorch, FastAPI, PostgreSQL, Streamlit).

[TABUĽKA: Použité technológie a ich úloha v systéme]


\section{Experimentálne výsledky a analýza} \label{vysledky}

\subsection{Výkonnostné porovnanie modelov} \label{}
Porovnanie modelov (BERT, GPT, LLaMA) na testovacom datasete.

[GRAF: Presnosť odpovedí podľa typu modelu]
\subsection{Analýza chýb a limitácií} \label{}
Identifikácia typov chýb (faktické, jazykové, sémantické).

[TABUĽKA: Príklady nesprávnych odpovedí a ich klasifikácia]
\subsection{Meranie používateľskej spokojnosti} \label{}
Výsledky dotazníkov a testov používateľov (študentov).

[GRAF: Hodnotenie spokojnosti používateľov (Likertova škála)]
\subsection{Diskusia výsledkov} \label{}
Interpretácia výsledkov, porovnanie s hypotézami, možné vysvetlenia zistení.


\section{Diskusia a odporúčania} \label{odporucania}

\subsection{Porovnanie s existujúcimi riešeniami} \label{}
Zhodnotenie prínosov EduGuide AI oproti iným AI asistentom vo vzdelávaní.

[TABUĽKA: Porovnanie funkcií EduGuide AI vs. konkurencia]
\subsection{Praktické implikácie a možnosti nasadenia} \label{}
Diskusia o potenciáli integrácie do praxe (napr. univerzitné portály, LMS).

[DIAGRAM: Možnosti nasadenia EduGuide AI v akademickej infraštruktúre]
\subsection{Limity výskumu a budúce smerovanie} \label{}
Identifikácia slabín projektu (obmedzenia dát, jazykové nuansy, etické aspekty) a návrhy na budúci výskum.


\section{Záver} \label{zaver}

\subsection{Zhrnutie hlavných prínosov práce} \label{}
Stručná rekapitulácia výsledkov, prínosov a praktických výstupov projektu EDUAI.
\subsection{Možnosti rozšírenia a budúce výzvy} \label{}
Návrhy na pokračovanie výskumu (napr. multimodálne modely, adaptívne učenie, personalizácia pre študentov).

%\acknowledgement{Ak niekomu chcete poďakovať\ldots}


% týmto sa generuje zoznam literatúry z obsahu súboru literatura.bib podľa toho, na čo sa v článku odkazujete
\bibliography{literatura}
\bibliographystyle{plain} % prípadne alpha, abbrv alebo hociktorý iný
\end{document}
